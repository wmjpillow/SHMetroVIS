% Introduction section is automatically added
Hello World. Please see \S\ref{sec:conclusion}.

\testfigure

\testtable

\section{Background}
Turn your annotated bibliography into a background section here.

\section{Method}
Describe your approach. 
Thoroughly describe all stimuli, major design descisions, and so on.

For ideas on what to include for your paper, refer to Tamara Munzner's Processes and Pitfalls paper.
In particular, look for examples of papers similar to yours in her paper, and mimic what they talk about.

\section{Results}
Describe your results here. Focus on the facts and statistics. Do not add your interpretation of the results here.

\section{Discussion}
A discussion typically starts with rehashing the main results, and speculating on the possible underlying causes of what you found.

Towards the end of the discussion, it's good to broaden your discussion to how your results relate to other major themes in the field.

\section{Conclusion}
\label{sec:conclusion}
Remind the reader of what the paper was about, what the main results were, and 1-2 things for future work.
